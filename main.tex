\documentclass{llncs}
\usepackage{graphicx}
\usepackage{parskip}
\usepackage{comment}
\usepackage{ulem}
\usepackage{float}
\usepackage[english]{babel}
\usepackage[utf8x]{inputenc}
\usepackage{amsmath}
\usepackage[colorinlistoftodos]{todonotes}
\usepackage{titletoc}
\usepackage{lipsum}
\usepackage{bibgerm}
\pagestyle{plain}

\begin{document}
\begin{titlepage}\centering
  
{\scshape\LARGE Universität Salzburg \par}
	\vspace{1cm}  
\begin{figure}
\centering
\includegraphics[width=0.2\textwidth]{uni_siegel.png}
\caption{\label{fig:uni_siegel}Abbildung Siegel der Universität}
\end{figure}
    
	
	{\scshape\Large Lehrveranstaltung: \\
    Natural Computation\par}
	\vspace{1cm}
	{\huge\bfseries \uline{Playing Style Analysis}\par}
	\vspace{0.5cm} 
    
    {\scshape\Large Team: \\
Diamonds \par}
	\vspace{1cm}
    
	{\Large\itshape Authoren:\\ Tabea Biel, Philipp Mayer, Vivien Wallner, Katharina Reiter \par}
	\vfill
	{\Large\itshape Lehrveranstaltungsleiter: \\
	Ao. Univ.-Prof. Dipl.-Ing. Dr. Helmut A. Mayer \\ \textsc{}}

	\vfill

	{\large \today\par}
\end{titlepage}

\begin{abstract}
Das Projekt \glqq Implementation of a general Poker Framework validated by Evolution of Computer Player Strategies\grqq, auf dem diese Arbeit basiert, befasst sich mit der Schaffung eines Poker-Frameworks, dass durch die Entwickulng von Poker Spielern mittels genetischer Programmierung bestimmt wird. Das Framework erlaubt es den Poker Spielern an Tunieren teilzunehmen und gegeneinander zu konkurieren. Dieses Projekt (\glqq Playing Style Analysis\grqq) konzentriert sich auf den Spielstil der einzelnen Spieler. Die Spieler werden auf Grund ihrer Fitness bewertet, welche der Qualität ihrer Strategien entspricht. Die solidesten Spieler werden sorgfältig ausgewählt und aussortiert, bis am Ende der passende Spieler übergeblieben ist. Der Schwerpunk dieses Projekts liegt in der Bereitstellung von Informationen über jeden Poker Spieler, um eine Weiterentwicklung des Spielers gewähren zu können. Dies geschieht durch die Festlegung, welche Informationen relevant sind, um sie in einer Spielstrategie verwenden zu können. Es ist nicht wichtig die Strategien zu verbessern, sondern den Spielern die Informationen zu geben, damit sie in den Spielstrategien besser handeln. In dieser Arbeit werden dann die einzelnen Spielstile analysiert und beschrieben.

\end{abstract}


\end{document}